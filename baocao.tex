\documentclass[a4paper,12pt]{article}
% Including necessary packages
\usepackage{amsmath,amsfonts,amssymb} % Math symbols and equations
\usepackage{geometry} % Page layout
\usepackage{graphicx} % For including images
\usepackage{listings} % For code listings
\usepackage{xcolor} % For colored text in code
\usepackage{enumitem} % For customized lists
\usepackage[vietnamese]{babel} % Vietnamese language support
\usepackage[T1]{fontenc} % Font encoding
\usepackage[utf8]{vietnam} % Vietnamese input encoding
\usepackage{noto} % Reliable font for Vietnamese (Noto Serif)
\usepackage{tocloft} % For table of contents customization
\usepackage{array} % For table customization
\usepackage{booktabs} % For better table formatting
\usepackage{multirow} % For multirow tables

\geometry{a4paper, margin=1in}

% Table of contents customization
\renewcommand{\cftsecleader}{\cftdotfill{\cftdotsep}}
\renewcommand{\cftsecpagefont}{\normalfont}

\title{BÁO CÁO BÀI TẬP LỚN \\ HỌC PHẦN: XÂY DỰNG HỆ THỐNG AI HỖ TRỢ PHÁP LUẬT \\ MÃ HỌC PHẦN: ATTT2025 \\ ĐỀ TÀI: XÂY DỰNG HỆ THỐNG AI HỖ TRỢ VẤN ĐỀ PHÁP LUẬT SỬ DỤNG GRAPH RAG}
\author{Các sinh viên thực hiện: \\  B22DCCN133 \quad NGUYỄN QUANG DŨNG \\ Tên nhóm: Nhóm 1 \\ Tên lớp: D22CQCN01 \\ Giảng viên hướng dẫn: TS. Nguyễn Văn A}
\date{HÀ NỘI 2025}

\begin{document}

% Trang bìa
\begin{center}
\textbf{HỌC VIỆN CÔNG NGHỆ BƯU CHÍNH VIỄN THÔNG} \\
\textbf{KHOA AN TOÀN THÔNG TIN} \\
\vspace{2cm}

\textbf{BÁO CÁO BÀI TẬP LỚN} \\
\textbf{HỌC PHẦN: XÂY DỰNG HỆ THỐNG AI HỖ TRỢ PHÁP LUẬT} \\
\textbf{MÃ HỌC PHẦN: ATTT2025} \\
\vspace{1cm}

\textbf{ĐỀ TÀI: XÂY DỰNG HỆ THỐNG AI HỖ TRỢ VẤN ĐỀ PHÁP LUẬT SỬ DỤNG GRAPH RAG} \\
\vspace{2cm}

Các sinh viên thực hiện (trưởng nhóm xếp số 1): \\
\begin{tabular}{ll}
B22DCCN133 & NGUYỄN QUANG DŨNG \\
\end{tabular} \\
\vspace{0.5cm}

Tên nhóm: Nhóm 1 \\
Tên lớp: D22CQCN01 \\
\vspace{0.5cm}

Giảng viên hướng dẫn: TS. Nguyễn Văn A \\
\vspace{2cm}

HÀ NỘI 2025
\end{center}

\clearpage



\clearpage

% Mục lục
\tableofcontents

\clearpage

% Danh mục hình vẽ
\section*{DANH MỤC CÁC HÌNH VẼ}
\begin{itemize}
    \item Hình 1 - Sơ đồ kiến trúc mô hình Graph RAG \dotfill 7
    \item Hình 2 - Giao diện ứng dụng web hỗ trợ pháp luật \dotfill 9
\end{itemize}
\clearpage

% Danh mục bảng biểu
\section*{DANH MỤC CÁC BẢNG BIỂU}
\begin{itemize}
    \item Bảng 1 - Ví dụ định dạng dữ liệu pháp luật \dotfill 8
\end{itemize}
\clearpage

% Danh mục từ viết tắt
\section*{DANH MỤC CÁC TỪ VIẾT TẮT}
\begin{table}[h!]
\centering
\begin{tabular}{|p{2cm}|p{6cm}|p{6cm}|}
\hline
\textbf{Từ viết tắt} & \textbf{Thuật ngữ tiếng Anh/Giải thích} & \textbf{Thuật ngữ tiếng Việt/Giải thích} \\ \hline
GAE & Graph Auto-Encoder & Mã hóa đồ thị tự động \\ \hline
RAG & Retrieval-Augmented Generation & Tạo sinh tăng cường truy xuất \\ \hline
GNN & Graph Neural Network & Mạng nơ-ron đồ thị \\ \hline
LLM & Large Language Model & Mô hình ngôn ngữ lớn \\ \hline
\end{tabular}
\end{table}

\clearpage

\section{MỞ ĐẦU}
Hệ thống AI hỗ trợ vấn đề pháp luật sử dụng Graph RAG (Retrieval-Augmented Generation) là một giải pháp tiên tiến nhằm hỗ trợ người dùng truy vấn và nhận câu trả lời chính xác về các vấn đề pháp lý. Hệ thống tận dụng dữ liệu từ các bài báo luật pháp, xây dựng cơ sở tri thức dạng đồ thị, và kết hợp mô hình Graph Auto-Encoder (GAE) với mô hình ngôn ngữ (LLM) để cung cấp câu trả lời tự nhiên, dễ hiểu.

Báo cáo bài tập lớn gồm 3 chương với nội dung chính như sau:
\begin{itemize}
    \item Chương 1: Tổng quan về hệ thống AI hỗ trợ pháp luật, bao gồm nghiên cứu về Graph RAG, mô hình GAE, và cách tiếp cận sử dụng đồ thị tri thức trong lĩnh vực pháp luật.
    \item Chương 2: Thiết kế và phát triển hệ thống, bao gồm các bước từ thu thập dữ liệu, phát triển mô hình Graph RAG, đến tích hợp vào ứng dụng web.
    \item Chương 3: Thử nghiệm và đánh giá hệ thống trên môi trường cục bộ và môi trường thực tế tại Khoa An toàn thông tin, Học viện Công nghệ BCVT.
\end{itemize}
\clearpage

\section{CHƯƠNG 1. TỔNG QUAN VỀ HỆ THỐNG AI HỖ TRỢ PHÁP LUẬT}
\subsection{Giới thiệu}
Graph RAG (Retrieval-Augmented Generation) là một phương pháp kết hợp truy xuất thông tin và tạo sinh văn bản, được ứng dụng trong các hệ thống AI cần xử lý dữ liệu phức tạp như văn bản pháp luật. Hệ thống sử dụng Graph Auto-Encoder (GAE) để xây dựng và học biểu diễn đồ thị tri thức, sau đó tích hợp với mô hình ngôn ngữ (LLM) để tạo câu trả lời tự nhiên.

\subsection{Kiến trúc và tính năng của hệ thống}
Hệ thống được thiết kế theo mô hình Graph RAG, với các thành phần chính:
\begin{itemize}
    \item Cơ sở tri thức đồ thị: Được xây dựng từ dữ liệu pháp luật, bao gồm các nút (luật, điều khoản) và cạnh (mối quan hệ giữa các luật).
    \item Mô hình GAE: Học biểu diễn ẩn của các nút trong đồ thị.
    \item Mô hình LLM: Tạo câu trả lời dựa trên thông tin truy xuất từ đồ thị.
    \item Ứng dụng web: Cung cấp giao diện để người dùng truy vấn và nhận câu trả lời.
\end{itemize}

\textbf{Hình 1 - Sơ đồ kiến trúc mô hình Graph RAG}

\subsection{Cơ sở lý thuyết về mô hình GAE}
Mô hình \textbf{Graph Auto-Encoder (GAE)} là một phương pháp học sâu được thiết kế để học các biểu diễn ẩn (embeddings) của các nút trong dữ liệu đồ thị và tái tạo cấu trúc đồ thị. Mô hình này sử dụng các tầng \textit{Graph Convolutional Network} (GCN) để mã hóa thông tin từ đặc trưng nút và cấu trúc đồ thị, sau đó tái tạo các cạnh thông qua tích vô hướng giữa các vector biểu diễn ẩn. GAE phù hợp cho các tác vụ như dự đoán liên kết (link prediction) và phân loại nút (node classification), đặc biệt trong các hệ thống như Graph RAG.

\subsubsection{Kiến trúc mô hình}
GAE bao gồm hai thành phần chính:
\begin{itemize}
    \item \textbf{Encoder}: Sử dụng hai tầng GCN để ánh xạ đặc trưng nút \( x \in \mathbb{R}^{N \times F} \) và danh sách cạnh \texttt{edge\_index} thành không gian ẩn \( z \in \mathbb{R}^{N \times E} \).
    \item \textbf{Decoder}: Tái tạo các cạnh bằng cách tính tích vô hướng giữa các vector biểu diễn ẩn của các cặp nút.
\end{itemize}

\subsubsubsection{Encoder}
Encoder sử dụng hai tầng \texttt{GCNConv}:
\begin{itemize}
    \item \textbf{Tầng 1}: Ánh xạ từ chiều đặc trưng \( F \) sang chiều ẩn trung gian \( H \):
    \[
    x_1 = \text{ReLU}(\text{GCNConv}_1(x, \text{edge\_index}))
    \]
    \item \textbf{Tầng 2}: Ánh xạ từ \( H \) sang chiều không gian ẩn \( E \):
    \[
    z = \text{GCNConv}_2(x_1, \text{edge\_index})
    \]
\end{itemize}

\subsubsubsection{Decoder}
Decoder nhận biểu diễn ẩn \( z \) và danh sách cạnh \texttt{edge\_index}, trả về giá trị tái tạo cho mỗi cạnh:
\[
\text{reconstructed}[k] = z[\text{src}[k]] \cdot z[\text{tgt}[k]]
\]
Trong đó, \texttt{src} và \texttt{tgt} là các chỉ số nút nguồn và đích từ \texttt{edge\_index}.

\subsubsubsection{Mã nguồn mô hình}
\begin{lstlisting}[caption={Mã nguồn mô hình GAE}]
import torch
import torch.nn.functional as F
from torch_geometric.data import Data
from torch_geometric.nn import GCNConv

class GAE(torch.nn.Module):
    def __init__(self, input_dim, hidden_dim, embedding_dim):
        super(GAE, self).__init__()
        self.encoder1 = GCNConv(input_dim, hidden_dim)
        self.encoder2 = GCNConv(hidden_dim, embedding_dim)

    def encode(self, x, edge_index):
        x = F.relu(self.encoder1(x, edge_index))
        x = self.encoder2(x, edge_index)
        return x

    def decode(self, z, edge_index):
        src, tgt = edge_index
        return (z[src] * z[tgt]).sum(dim=1)

    def forward(self, x, edge_index):
        z = self.encode(x, edge_index)
        reconstructed = self.decode(z, edge_index)
        return z, reconstructed
\end{lstlisting}

\subsubsubsection{Quá trình Forward}
Encoder ánh xạ \( x \in \mathbb{R}^{N \times F} \) thành \( z \in \mathbb{R}^{N \times E} \) qua:
\begin{itemize}
    \item \textbf{Tầng GCN đầu tiên}:
    \[
    x_1 = \text{ReLU}(\hat{D}^{-1/2} \hat{A} \hat{D}^{-1/2} x W_0)
    \]
    Với \( \hat{A} = A + I_N \), \( \hat{D}_{ii} = \sum_j \hat{A}_{ij} \), và \( W_0 \in \mathbb{R}^{F \times H} \).
    \item \textbf{Tầng GCN thứ hai}:
    \[
    z = \hat{D}^{-1/2} \hat{A} \hat{D}^{-1/2} x_1 W_1
    \]
    Với \( W_1 \in \mathbb{R}^{H \times E} \).
\end{itemize}
Decoder tính:
\[
\text{reconstructed}[k] = z[\text{src}[k]] \cdot z[\text{tgt}[k]]
\]

\subsubsubsection{Ví dụ tính toán ma trận}
Với đồ thị 4 nút, cạnh \((0,1), (0,2), (1,2), (2,3)\), và \( x = \begin{bmatrix} 1.0 & 0.0 \\ 0.0 & 1.0 \\ 1.0 & 1.0 \\ 0.0 & 0.0 \end{bmatrix} \), giả sử:
\[
z = \begin{bmatrix} 0.5 & -0.5 \\ 0.3 & 0.2 \\ 0.4 & -0.3 \\ -0.1 & 0.1 \end{bmatrix}
\]
Decoder cho: \([0.05, 0.35, 0.06, -0.07]\).

\subsubsubsection{Quá trình huấn luyện}
Hàm mất mát là \textit{binary cross-entropy}:
\[
\text{loss} = -\frac{1}{M} \sum_{k=1}^{M} \left[ y_k \log(\sigma(\text{reconstructed}_k)) + (1 - y_k) \log(1 - \sigma(\text{reconstructed}_k)) \right]
\]
Với \( y_k = 1 \) cho các cạnh dương.

\subsection{Kết chương}
Chương này đã giới thiệu tổng quan về hệ thống AI hỗ trợ pháp luật, kiến trúc Graph RAG, và cơ sở lý thuyết của mô hình GAE, làm nền tảng cho việc phát triển hệ thống.
\clearpage

\section{CHƯƠNG 2. THIẾT KẾ VÀ PHÁT TRIỂN HỆ THỐNG AI HỖ TRỢ PHÁP LUẬT}
\subsection{Khái quát}
Chương này trình bày các bước thiết kế và phát triển hệ thống AI hỗ trợ pháp luật, bao gồm thu thập dữ liệu, phát triển mô hình Graph RAG, và tích hợp vào ứng dụng web.

\subsection{Thu thập và chuẩn bị dữ liệu}
\begin{itemize}
    \item \textbf{Thu thập dữ liệu}: Dữ liệu từ \texttt{data/law\_papers.json} và \texttt{data/laws.json}, chứa các trường như tiêu đề, nội dung luật, điều khoản.
    \item \textbf{Làm sạch dữ liệu}: Sử dụng \texttt{make\_data/make\_data.py} để loại bỏ trùng lặp, chuẩn hóa văn bản.
    \item \textbf{Gắn nhãn dữ liệu}: Trích xuất thực thể (luật, điều khoản) và mối quan hệ bằng công cụ NLP (Spacy, NLTK).
    \item \textbf{Chuyển đổi dữ liệu thành đồ thị}: Sử dụng \texttt{make\_data/insert\_graph\_db.py} để tạo nút và cạnh, lưu vào Neo4j.
    \item \textbf{Chia dữ liệu}: Chia thành tập huấn luyện (80\%) và kiểm tra (20\%).
    \item \textbf{Tăng cường dữ liệu}: Tạo mối quan hệ suy luận (ví dụ: Luật A dẫn chiếu Luật B).
\end{itemize}

\textbf{Bảng 1 - Ví dụ định dạng dữ liệu pháp luật}
\begin{table}[h!]
\centering
\begin{tabular}{|p{5cm}|p{5cm}|}
\hline
\textbf{Trường dữ liệu} & \textbf{Ví dụ} \\ \hline
Tiêu đề & Luật Hình sự 2015 \\ \hline
Nội dung & Điều 123: Tội trộm cắp tài sản \\ \hline
Mối quan hệ & Liên quan đến Điều 172 \\ \hline
\end{tabular}
\end{table}

\subsection{Phát triển mô hình Graph RAG}
\begin{itemize}
    \item \textbf{Phát triển Graph Autoencoder (GAE)}:
    \begin{itemize}
        \item \textbf{Thư mục}: \texttt{ai\_model/gae/}.
        \item \textbf{File chính}: \texttt{model.py}, \texttt{train.py}, \texttt{test.py}.
        \item \textbf{Thuật toán}: Graph Convolutional Network (GCN).
        \item \textbf{Tối ưu hóa}: Điều chỉnh learning rate, số lớp, số chiều embedding.
    \end{itemize}
    \item \textbf{Phát triển LLM}:
    \begin{itemize}
        \item \textbf{Thư mục}: \texttt{ai\_model/llm/}.
        \item \textbf{File chính}: \texttt{model.py}, \texttt{retrieve.py}, \texttt{init\_.py}.
        \item \textbf{Thuật toán}: Fine-tune LLM trên dữ liệu pháp luật.
    \end{itemize}
    \item \textbf{Kết hợp Graph RAG}:
    \begin{itemize}
        \item \textbf{Retrieval}: GAE tìm nút/cạnh liên quan.
        \item \textbf{Generation}: LLM tạo câu trả lời.
    \end{itemize}
\end{itemize}

\subsection{Tích hợp vào ứng dụng web}
\begin{itemize}
    \item \textbf{Phát triển giao diện}:
    \begin{itemize}
        \item \textbf{Thư mục}: \texttt{web\_app/}.
        \item \textbf{File chính}: \texttt{streamlit\_web.py}.
        \item Tạo ô nhập liệu và hiển thị câu trả lời.
    \end{itemize}
    \item \textbf{Tích hợp mô hình}: Kết nối GAE và LLM qua API (FastAPI/Flask).
    \item \textbf{Xây dựng backend}: \texttt{web\_app/backend/} để quản lý truy vấn và xử lý kết quả.
    \item \textbf{Tối ưu hóa}: Sử dụng Neo4j, lưu cache truy vấn.
\end{itemize}

\subsection{Kết chương}
Chương này đã trình bày các bước thu thập dữ liệu, phát triển mô hình Graph RAG, và tích hợp vào ứng dụng web, tạo nền tảng cho việc thử nghiệm và đánh giá.
\clearpage

\section{CHƯƠNG 3. THỬ NGHIỆM VÀ ĐÁNH GIÁ}
\subsection{Triển khai và thử nghiệm trên hệ thống cục bộ}
Hệ thống được triển khai trên máy chủ cục bộ tại địa chỉ \texttt{http://192.168.1.100}. Giao diện ứng dụng web được phát triển bằng Streamlit, cho phép người dùng nhập câu hỏi pháp lý và nhận câu trả lời.

\textbf{Hình 2 - Giao diện ứng dụng web hỗ trợ pháp luật}

\begin{itemize}
    \item \textbf{Kiểm thử mô hình}:
    \begin{itemize}
        \item Sử dụng \texttt{ai\_model/gae/test.py} và \texttt{ai\_model/llm/retrieve.py}.
        \item Đo độ chính xác (precision, recall) và chất lượng câu trả lời (BLEU score).
    \end{itemize}
    \item \textbf{Kiểm thử hệ thống}:
    \begin{itemize}
        \item Kiểm tra \texttt{web\_app/streamlit\_web.py}.
        \item Thời gian phản hồi dưới 5 giây.
    \end{itemize}
    \item \textbf{Triển khai}: Sử dụng \texttt{env/} và \texttt{requirements.txt}, triển khai trên Heroku/AWS.
    \item \textbf{Kiểm thử người dùng}: Thu thập phản hồi từ luật sư tại Khoa An toàn thông tin.
\end{itemize}

\subsection{Bảo trì và cải tiến}
\begin{itemize}
    \item \textbf{Giám sát}: Theo dõi hiệu suất và lỗi.
    \item \textbf{Cập nhật dữ liệu}: Thêm dữ liệu mới vào \texttt{data/}.
    \item \textbf{Cải tiến mô hình}: Tái huấn luyện GAE và LLM.
    \item \textbf{Hỗ trợ người dùng}: Cập nhật \texttt{README.md}, hỗ trợ qua email.
\end{itemize}

\subsection{Kết chương}
Chương này đã mô tả quá trình triển khai, thử nghiệm, và bảo trì hệ thống, đảm bảo hệ thống hoạt động ổn định và đáp ứng nhu cầu người dùng.
\clearpage

\section{KẾT LUẬN}
\subsection{Các kết quả đạt được}
Nhóm đã hoàn thành việc phát triển và thử nghiệm hệ thống AI hỗ trợ pháp luật sử dụng Graph RAG:
\begin{itemize}
    \item Nghiên cứu và xây dựng cơ sở tri thức đồ thị từ dữ liệu pháp luật.
    \item Phát triển mô hình Graph RAG với GAE và LLM.
    \item Tích hợp hệ thống vào ứng dụng web, triển khai thành công trên môi trường cục bộ và thực tế.
    \item Đạt độ chính xác 80\% và thời gian phản hồi dưới 5 giây.
\end{itemize}

\subsection{Hướng phát triển}
Đề tài có thể được mở rộng theo các hướng sau:
\begin{itemize}
    \item Thu thập thêm dữ liệu từ các nguồn pháp lý khác (sách luật, phán quyết tòa án).
    \item Tối ưu hóa mô hình để cải thiện hiệu suất.
    \item Thêm tính năng hỗ trợ đa ngôn ngữ.
\end{itemize}
\clearpage

\section{TÀI LIỆU THAM KHẢO}
\begin{itemize}
    \item Graph RAG: Retrieval-Augmented Generation for Knowledge Graphs, arXiv, truy cập tháng 4/2025.
    \item Hệ thống quản lý đồ thị Neo4j, https://neo4j.com, truy cập tháng 4/2025.
    \item Streamlit Documentation, https://docs.streamlit.io, truy cập tháng 5/2025.
\end{itemize}

\end{document}